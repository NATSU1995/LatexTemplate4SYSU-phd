% !Mode:: "TeX:UTF-8"

% 定义编译方式 dvipdfmx 或者 pdflatex,默认为 dvipdfmx
% 方式编译,如果需要修改,只需改变花括号中的内容即可。
\def\usewhat{pdflatex}

% book作为文档类
% 插入空白页可以设置openright
\documentclass[12pt,openany,oneside]{book}


% !Mode:: "TeX:UTF-8"

% 原作者信息Original Authors:
% 张井   Jing Zhang: prayever@gmail.com     天津大学2010级管理与经济学部信息管理与信息系统专业硕士生
% 余蓝涛 Lantao Yu: lantaoyu1991@gmail.com  天津大学2008级精密仪器与光电子工程学院测控技术与仪器专业本科生

% 中大移植版by林海涛&徐浩晖
% 中大博士论文版by蔡芸

%%%%%%%%%% Package %%%%%%%%%%%%

% 支持插图处理
\usepackage{graphicx}
\usepackage[a4paper,text={146.4true mm,239.2 true mm},top= 26.2true mm,left=31.8 true mm,head=6true mm,headsep=6.5true mm,foot=16.5true mm]{geometry}
\usepackage[acronym]{glossaries}
% 支持版面尺寸设置
% 支持国际标准单位
\usepackage[squaren]{SIunits}               

\usepackage{titlesec}                       % 控制标题的宏包
\usepackage{titletoc}                       % 控制目录的宏包
\usepackage{fancyhdr}                       % fancyhdr宏包 支持页眉和页脚的相关定义
\usepackage[UTF8]{ctex}                     % 支持中文显示
\usepackage{CJKpunct}                       % 精细调整中文的标点符号
\usepackage{color}                          % 支持彩色
%\usepackage{amsthm} 
\usepackage{mathrsfs}
\usepackage{amsmath}                        % AMSLaTeX宏包 用来排出更加漂亮的公式
\usepackage{amssymb}                        % 数学符号生成命令
\usepackage{float}
\usepackage[below]{placeins}    %允许上一个section的浮动图形出现在下一个section的开始部分,还提供\FloatBarrier命令,使所有未处理的浮动图形立即被处理
\usepackage{multirow}                       % 使用Multirow宏包,使得表格可以合并多个row格
\usepackage{booktabs}                       % 表格,横的粗线;\specialrule{1pt}{0pt}{0pt}
\usepackage{longtable}                      % 支持跨页的表格。
\usepackage{tabularx}                       % 自动设置表格的列宽
\usepackage{subfigure}                      % 支持子图 %centerlast 设置最后一行是否居中
\usepackage[subfigure]{ccaption}            % 支持子图的中文标题
\usepackage[sort&compress,numbers]{natbib}  % 支持引用缩写的宏包
\newcommand{\upcite}[1]{\textsuperscript{\citep{#1}}}
\usepackage{enumitem}                       % 使用enumitem宏包,改变列表项的格式
\usepackage{calc}                           % 长度可以用+ - * / 进行计算
\usepackage{txfonts}                        % 字体宏包
\usepackage{bm}                             % 处理数学公式中的黑斜体的宏包
\usepackage[amsmath,thmmarks,hyperref]{ntheorem}  % 定理类环境宏包,其中 amsmath 选项用来兼容 AMS LaTeX 的宏包
\usepackage{CJKnumb}                        % 提供将阿拉伯数字转换成中文数字的命令
\usepackage{indentfirst}                    % 首行缩进宏包
\usepackage{CJKutf8}                        % 用在UTF8编码环境下,它可以自动调用CJK,同时针对UTF8编码作了设置
%\usepackage{hypbmsec}                      % 用来控制书签中标题显示内容
\newcommand{\tabincell}[2]{\begin{tabular}{@{}#1@{}}#2\end{tabular}}
\usepackage{xcolor}
%支持代码环境
\usepackage{listings}
\lstset{numbers=left,
language=[ANSI]{C},
numberstyle=\tiny,
extendedchars=false,
showstringspaces=false,
breakatwhitespace=false,
breaklines=true,
captionpos=b,
keywordstyle=\color{blue!70},
commentstyle=\color{red!50!green!50!blue!50},
frame=shadowbox,
rulesepcolor=\color{red!20!green!20!blue!20}
}
%支持算法环境
\usepackage[boxed,ruled,lined]{algorithm2e}
%\usepackage[ruled]{algorithm2e}                 %算法排版样式1
%\usepackage[ruled,vlined]{algorithm2e}          %算法排版样式2
%\usepackage[linesnumbered,boxed]{algorithm2e}   %算法排版样式3
\usepackage{algorithmic}
%\usepackage{mathrsfs}
\usepackage{array}
\newcommand{\PreserveBackslash}[1]{\let\temp=\\#1\let\\=\temp}
\newcolumntype{C}[1]{>{\PreserveBackslash\centering}p{#1}}
\newcolumntype{R}[1]{>{\PreserveBackslash\raggedleft}p{#1}}
\newcolumntype{L}[1]{>{\PreserveBackslash\raggedright}p{#1}}
\setcounter{tocdepth}{4}
\setcounter{secnumdepth}{3}


% 定义本文所使用宏包

\usepackage{nomencl}
% 更改术语表标题并居中显示
\renewcommand{\nomname}{\makebox[\linewidth]{缩写、符号清单、术语表}}

% 方法1中文术语表(不能自动换行)
%\newcommand{\nomchinese}[1]
%{
%	\renewcommand{\nomentryend}{\hspace*{\fill}\makebox[4.5cm][l]{#1}}
%}

% 方法2中文术语表
\newcommand{\nomdescr}[1]{
    \parbox[t]{8cm}{\raggedright\strut #1 \strut}
}
\newcommand{\nomdescrchn}[1]{
    \hfill\parbox[t]{4.5cm}{\strut #1 \strut}\ignorespaces
}
\newcommand{\nomchn}[4][]{
    \nomenclature[#1]{#2}{
        \nomdescr{#3}
        \nomdescrchn{#4}
    }
}



% 生成有书签的 pdf 及其生成方式。通常可以在 tjumain.tex 文件的第一行选择 pdflatex 或者是 dvipdfmx 编译手段。如果选择前者,则使用 pdflatex + pdflatex 编译; 如果选择后者,在编译的时候选择 latex + bibtex + latex + latex 编译。出现混淆的时候,系统会报错。
% 如果您的pdf制作中文书签有乱码使用如下命令,就可以解决了
\def\atemp{dvipdfmx}\ifx\atemp\usewhat
\usepackage[dvipdfmx,unicode,               % dvipdfmx 编译, 加入了中文复制,粘贴支持引擎。
            pdfstartview=FitH,
            bookmarksnumbered=true,
            bookmarksopen=true,
            colorlinks=false,
            pdfborder={0 0 1},
            citecolor=blue,
            linkcolor=red,
            anchorcolor=green,
            urlcolor=blue,
            breaklinks=true
            ]{hyperref}
\fi

\def\atemp{pdflatex}\ifx\atemp\usewhat
\usepackage{cmap}                           % pdflatex 编译时,可以生成可复制、粘贴的中文 PDF 文档, 缺点是在Windows上显示时效果不大好,字体发虚
\usepackage[pdftex,unicode,
            %CJKbookmarks=true,
            bookmarksnumbered=true,
            bookmarksopen=true,				
            hidelinks
%            colorlinks=true,		% original false
%            pdfborder={0 0 1},
%            citecolor=black,		% original blue
%            linkcolor=black,		% original red
%            anchorcolor=black,		% original green
%            urlcolor=blue,
%            breaklinks=false		% original true
            ]{hyperref}
\fi

% 新增for url
\usepackage{url}


% 定义所有的.eps/.pdf图片文件在figures子目录下
\graphicspath{{figures/}}


% 生成术语表
%\makenomenclature


\begin{document}

% 开始中文字体使用
\begin{CJK*}{UTF8}{song}

% 完成对论文各个部分格式的设置
% !Mode:: "TeX:UTF-8"

%%%%%%%%%%%%%%%%% Fonts Definition and Basics %%%%%%%%%%%%%%%%%
\newcommand{\song}{\CJKfamily{song}}    % 宋体
\newcommand{\fs}{\CJKfamily{fs}}        % 仿宋体
\newcommand{\kai}{\CJKfamily{kai}}      % 楷体
\newcommand{\hei}{\CJKfamily{hei}}      % 黑体
\newcommand{\li}{\CJKfamily{li}}        % 隶书
\newcommand{\chuhao}{\fontsize{28pt}{28pt}\selectfont}       % 初号, 单倍行距
\newcommand{\yihao}{\fontsize{26pt}{26pt}\selectfont}       % 一号, 单倍行距
\newcommand{\xiaoyi}{\fontsize{24pt}{24pt}\selectfont}      % 小一, 单倍行距
\newcommand{\erhao}{\fontsize{22pt}{1.25\baselineskip}\selectfont}       % 二号, 1.25倍行距
\newcommand{\xiaoer}{\fontsize{18pt}{18pt}\selectfont}      % 小二, 单倍行距
\newcommand{\sanhao}{\fontsize{16pt}{16pt}\selectfont}      % 三号, 单倍行距
\newcommand{\xiaosan}{\fontsize{15pt}{15pt}\selectfont}     % 小三, 单倍行距
\newcommand{\sihao}{\fontsize{14pt}{14pt}\selectfont}       % 四号, 单倍行距
\newcommand{\sihaot}{\fontsize{14pt}{20pt}\selectfont}       % 四号, 1.25行距
\newcommand{\xiaosi}{\fontsize{12pt}{12pt}\selectfont}      % 小四, 单倍行距
\newcommand{\wuhao}{\fontsize{10.5pt}{10.5pt}\selectfont}   % 五号, 单倍行距
\newcommand{\xiaowu}{\fontsize{9pt}{9pt}\selectfont}        % 小五, 单倍行距

% 重新定义了波浪符~的意义
\CJKtilde

% 定义章的pre-post名称
\newcommand\prechaptername{第}
\newcommand\postchaptername{章}

% 调整中文字符的表示,行内占一个字符宽度,行尾占半个字符宽度
% 行末半角?
\punctstyle{hangmobanjiao}             

% 调整罗列环境的布局
\setitemize{leftmargin=3em,itemsep=0em,partopsep=0em,parsep=0em,topsep=-0em}
\setenumerate{leftmargin=3em,itemsep=0em,partopsep=0em,parsep=0em,topsep=0em}

% 避免宏包 hyperref 和 arydshln 不兼容带来的目录链接失效的问题。
\def\temp{\relax}
\let\temp\addcontentsline
\gdef\addcontentsline{\phantomsection\temp}

% 自定义项目列表标签及格式 \begin{publist} 列表项 \end{publist}
\newcounter{pubctr} %自定义新计数器
\newenvironment{publist}{%%%%%定义新环境
\begin{list}{[\arabic{pubctr}]} %%标签格式
    {
     \usecounter{pubctr}
     \setlength{\leftmargin}{2em}   % 左边界 \leftmargin =\itemindent + \labelwidth + \labelsep
     \setlength{\itemindent}{0em}     % 标号缩进量
     \setlength{\labelsep}{1em}       % 标号和列表项之间的距离,默认0.5em
     \setlength{\rightmargin}{0em}    % 右边界
     \setlength{\topsep}{0ex}         % 列表到上下文的垂直距离
     \setlength{\parsep}{0ex}         % 段落间距
     \setlength{\itemsep}{0ex}        % 标签间距
     \setlength{\listparindent}{0pt}  % 段落缩进量
    }}
{\end{list}}

\makeatletter
	\renewcommand\normalsize{
		\@setfontsize\normalsize{10.5pt}{10.5pt} % 小四对应 12 pt
		\setlength\abovedisplayskip{4pt}
		\setlength\abovedisplayshortskip{4pt}
		\setlength\belowdisplayskip{\abovedisplayskip}
		\setlength\belowdisplayshortskip{\abovedisplayshortskip}
		\let\@listi\@listI}
	
	% 不同的行距设置
	% TJU原始值1.63
	% 设为1.8则一页31行,1.95则一页29行(目前采用值)
	\def\defaultfont{\renewcommand{\baselinestretch}{1.8}\normalsize\selectfont} % 设置行距,正文一页29行
	
	% 控制字间距,使每行 34 个汉字
	\renewcommand{\CJKglue}{\hskip -0.1 pt plus 0.08\baselineskip} 
\makeatother

%%%%%%%%%%%%% Contents 目录 %%%%%%%%%%%%%%%%%

\renewcommand{\contentsname}{目\qquad 录}

% 控制目录深度,改为1
\setcounter{tocdepth}{1}

\titlecontents{chapter}[2em]{\vspace{.0\baselineskip}\sihao\song}	% 可以重调skip
	{\prechaptername~\thecontentslabel~\postchaptername\quad}{}
	{\!\titlerule*[5pt]{$\cdot$}\!\!\!\!\sihao\contentspage}	% 调整点的距离
\titlecontents{section}[3em]{\vspace{-0.1\baselineskip}\xiaosi\song}
	{\thecontentslabel\quad}{}
	{\!\titlerule*[5pt]{$\cdot$}\!\!\!\!\xiaosi\contentspage}
\titlecontents{subsection}[4em]{\vspace{-0.2\baselineskip}\wuhao\song}
	{\thecontentslabel\quad}{}
	{\!\titlerule*[5pt]{$\cdot$}\!\!\!\!\wuhao\contentspage}
             
%%%%%%%%%% Chapter and Section 章节 %%%%%%%%%%%%%

\setcounter{secnumdepth}{4}
\setlength{\parindent}{2em}

% 如果使用第“一”章
%\renewcommand{\chaptername}{\prechaptername\CJKnumber{\thechapter}\postchaptername}
% 使用第“1”章
\renewcommand{\chaptername}{\prechaptername~\thechapter~\postchaptername}

% 此处修改的chapter title会被主文件定义覆盖
% chapter标题格式:小二,黑体,居中
\titleformat{\chapter}{\centering\sanhao\hei}{\chaptername}{2em}{}
\titlespacing{\chapter}{0pt}{0.1\baselineskip}{0.8\baselineskip}

% section标题格式:小三,宋体加粗,左对齐
\titleformat{\section}{\xiaosan\song\bfseries}{\thesection}{1em}{}
\titlespacing{\section}{0pt}{0.15\baselineskip}{0.25\baselineskip}

% subsection标题格式:四号,宋体加粗,左对齐
\titleformat{\subsection}{\sihao\song\bfseries}{\thesubsection}{1em}{}
\titlespacing{\subsection}{0pt}{0.1\baselineskip}{0.3\baselineskip}

% subsubsection标题格式:小四,宋体加粗,左对齐
\titleformat{\subsubsection}{\xiaosi\song\bfseries}{\thesubsubsection}{1em}{}
\titlespacing{\subsubsection}{0pt}{0.05\baselineskip}{0.1\baselineskip}

%%%%%%%%%% Table, Figure and Equation 图/表/公式 %%%%%%%%%%%%%%%%%

\renewcommand{\tablename}{表}
\renewcommand{\figurename}{图}

% 使图编号为 7-1 的格式
\renewcommand{\thefigure}{\arabic{chapter}-\arabic{figure}}

% 使子图编号为 a) 的格式
%\renewcommand{\thesubfigure}{\alph{subfigure})}
% 使子图编号为 (a) 的格式
\renewcommand{\thesubfigure}{(\alph{subfigure})}

% 使子表编号为 (a) 的格式
\renewcommand{\thesubtable}{(\alph{subtable})}
% 使表编号为 7-1 的格式
\renewcommand{\thetable}{\arabic{chapter}-\arabic{table}}
% 使公式编号为 7-1 的格式
\renewcommand{\theequation}{\arabic{chapter}-\arabic{equation}}

\makeatletter
	% 使子图引用也是7-1a)或7-1(a)的形式
	\renewcommand{\p@subfigure}{\thefigure}
\makeatother

% 定制浮动图形和表格标题样式
\makeatletter
	\long\def\@makecaption#1#2{
	   \vskip\abovecaptionskip
	   \sbox\@tempboxa{\centering\wuhao\song{#1\quad #2}}
	   \ifdim \wd\@tempboxa >\hsize
	     \centering\wuhao\song{#1\quad #2} \par	% narrower
	   \else
	     \global \@minipagefalse
	     \hb@xt@\hsize{\hfil\box\@tempboxa\hfil}
	   \fi
	   \vskip\belowcaptionskip}
\makeatother

% 用来控制longtable表头分隔符
\captiondelim{~~~~} 

%%%%%%%%%% Theorem Environment 定理 %%%%%%%%%%%%%%%%%
\theoremstyle{plain}
\theorembodyfont{\song\rmfamily}
\theoremheaderfont{\hei\rmfamily}
\newtheorem{theorem}{定理~}[chapter]
\newtheorem{lemma}{引理~}[chapter]
\newtheorem{axiom}{公理~}[chapter]
\newtheorem{proposition}{命题~}[chapter]
\newtheorem{prop}{性质~}[chapter]
\newtheorem{corollary}{推论~}[chapter]
\newtheorem{conclusion}{结论~}[chapter]
\newtheorem{definition}{定义~}[chapter]
\newtheorem{conjecture}{猜想~}[chapter]
\newtheorem{example}{例~}[chapter]
\newtheorem{remark}{注~}[chapter]
\newtheorem{assumption}{假设~}[chapter]
%\newtheorem{algorithm}{算法~}[chapter]
\newenvironment{proof}{\noindent{证明:}}{\hfill $ \square $ \vskip 4mm}
\theoremsymbol{$\square$}

%%%%%%%%%% Page: number, header and footer 页面设置 %%%%%%%%%%%%%%%%%

%\frontmatter 或 \pagenumbering{roman}
%\mainmatter 或 \pagenumbering{arabic}
\makeatletter
	\renewcommand\frontmatter{\clearpage
		\@mainmatterfalse}
\makeatother

%%%%%%%%%%%% References 参考文献 %%%%%%%%%%%%%%%%%

\renewcommand{\bibname}{参考文献}
% 重定义参考文献样式,来自thu
\makeatletter
\renewenvironment{thebibliography}[1]{
    %\titleformat{\chapter}{\raggedright\sihao\hei}{\chaptername}{2em}{}
    %\titleformat{\chapter}{\centering\sihao\hei}{\chaptername}{2em}{}
    %\titleformat{\chapter}{\centering\xiaoer\hei}{\chaptername}{2em}{}
   \chapter*{\bibname}
   \wuhao
   \list{\@biblabel{\@arabic\c@enumiv}}
        {\renewcommand{\makelabel}[1]{##1\hfill}
         \settowidth\labelwidth{0 cm}
         \setlength{\labelsep}{0pt}
         \setlength{\itemindent}{0pt}
         \setlength{\leftmargin}{\labelwidth+\labelsep}
         \addtolength{\itemsep}{-0.7em}
%         \addtolength{\itemsep}{-1.0em}
         \linespread{1.5}\selectfont	% 调整每个参考文献项内的间距 !!!
         \usecounter{enumiv}
         \let\p@enumiv\@empty
         \renewcommand\theenumiv{\@arabic\c@enumiv}}
    \sloppy\frenchspacing
    \clubpenalty4000
    \@clubpenalty \clubpenalty
    \widowpenalty4000
    \interlinepenalty4000
    \sfcode`\.\@m}
   {\def\@noitemerr
     {\@latex@warning{Empty `thebibliography' environment}}
    \endlist\frenchspacing}
\makeatother

% 缩小参考文献间的垂直间距
\addtolength{\bibsep}{-0.5em}

% 每个条目自第二行起缩进的距离
\setlength{\bibhang}{2em}

% 参考文献引用作为上标出现
\makeatletter
	\def\@cite#1#2{\textsuperscript{[{#1\if@tempswa , #2\fi}]}}
\makeatother

% 引用格式
\bibpunct{[}{]}{,}{s}{}{,}

%%%%%%%%%%%% Cover 封面、摘要、版权、致谢格式定义 %%%%%%%%%%%%%%%%%
 
\makeatletter % 一直到结尾

\def\ctitle#1{\def\@ctitle{#1}}\def\@ctitle{}
\def\etitle#1{\def\@etitle{#1}}\def\@etitle{}
\def\csubject#1{\def\@csubject{#1}}\def\@csubject{}
\def\esubject#1{\def\@esubject{#1}}\def\@esubject{}
\def\cauthor#1{\def\@cauthor{#1}}\def\@cauthor{}
\def\eauthor#1{\def\@eauthor{#1}}\def\@eauthor{}
\def\csupervisor#1{\def\@csupervisor{#1}}\def\@csupervisor{}
\def\esupervisor#1{\def\@esupervisor{#1}}\def\@esupervisor{}
\def\cdate#1{\def\@cdate{#1}}\def\@cdate{}
\long\def\cabstract#1{\long\def\@cabstract{#1}}\long\def\@cabstract{}
\long\def\eabstract#1{\long\def\@eabstract{#1}}\long\def\@eabstract{}
\def\ckeywords#1{\def\@ckeywords{#1}}\def\@ckeywords{}
\def\ekeywords#1{\def\@ekeywords{#1}}\def\@ekeywords{}
\def\cheading#1{\def\@cheading{#1}}\def\@cheading{}
\def\department#1{\def\@department{#1}}\def\@department{}
\def\direction#1{\def\@direction{#1}}\def\@direction{}

\pagestyle{fancy}
  \fancyhf{}
  %\fancyhead[C]{\song\wuhao \@cheading}  % 页眉
  \lhead{\song\wuhao \@cheading}  % 左页眉
%  \rhead{\prechaptername\CJKnumber{\thechapter}\postchaptername}    % 右页眉
  \rhead{\prechaptername~\thechapter~\postchaptername}    % 右页眉
  \fancyfoot[C]{\song\xiaowu ~\thepage~}
\newlength{\@title@width}

% 定义封面
\def\makecover{
   \phantomsection
    \pdfbookmark[-1]{\@ctitle}{ctitle}

\begin{titlepage}
\vspace*{31.5pt}
\begin{center}

  \vspace*{18pt}
  \hei\chuhao{\textbf{中山大学博士学位论文}}

  \vspace*{40pt}
  \song\xiaoer{\@ctitle}

  \xiaoer{\textrm{\@etitle}}
 
   \vspace*{40pt}
   \setlength{\@title@width}{6cm}	% 控制封面中下划线的长度。
   {\sihaot\song{
   \begin{tabular}{lc}
     姓~~~~~~名:  &  \underline{\makebox[\@title@width][c]{\@cauthor}} \\
     系~~~~~~别:   &  
     \underline{\makebox[\@title@width][c]{\@department}} \\
     专业名称:   &  \underline{\makebox[\@title@width][c]{\@csubject}}\\
     研究方向:   &  \underline{\makebox[\@title@width][c]{\@direction}}\\
     指导老师:   &  \underline{\makebox[\@title@width][c]{\@csupervisor}} \\
   \end{tabular}}
  }
  
  \vspace*{60pt}
   \setlength{\@title@width}{5cm}
   {\sihaot\song{
   \begin{tabular}{lc}
     答辩委员会主席:  &  \underline{\makebox[\@title@width][c]{~}} \\
     答辩委员会委员:  &  \underline{\makebox[\@title@width][c]{~}} \\
     ~ &  \underline{\makebox[\@title@width][c]{~}}\\
     ~ &  \underline{\makebox[\@title@width][c]{~}}\\
     ~ &  \underline{\makebox[\@title@width][c]{~}}\\
   \end{tabular}}	% 不加粗
  }

  \vspace*{60pt}

\song\sanhao{\@cdate}
\end{center}
\end{titlepage}


%%%%%%%%%%%%%%%%%%%   Originality Statement  %%%%%%%%%%%%%%%%%%%%%%%
\clearpage
\pdfbookmark[0]{论文原创性声明}{originality}
\chapter*{\centering\sanhao\song\bfseries 论文原创性声明}
\song\defaultfont
本人郑重声明:所呈交的学位论文,是本人在导师的指导下,独立进行研究工作所取得的成果。除文中已经注明引用的内容外,本论文不包含任何其他个人或集体已经发表或撰写过的作品成果。对本文的研究作出重要贡献的个人和集体,均已在文中以明确方式标明。本人完全意识到本声明的法律结果由本人承担。

\vspace*{40pt}
\begin{flushright}
\setlength{\@title@width}{5cm}
  {\sihao\song{
  \begin{tabular}{lc}
    学位论文作者签名:           &  \underline{\makebox[\@title@width][c]{~}} \\
    \qquad\qquad\qquad 日~~期:  &  \underline{\makebox[\@title@width][c]{~}} \\
  \end{tabular}}
 }
\end{flushright}

%%%%%%%%%%%%%%%%%%%   Authorization Statement  %%%%%%%%%%%%%%%%%%%%%%%
\vspace*{60pt}
\pdfbookmark[0]{学位论文使用授权声明}{authorization}
\begin{center}
  \sanhao\song\bfseries{学位论文使用授权声明}
\end{center}

\song\defaultfont
本人完全了解中山大学有关保留、使用学位论文的规定,即:学校有权保留学位论文并向国家主管部门或其指定机构送交论文的电子版和纸质版,有权将学位论文用于非赢利目的的少量复制并允许论文进入学校图书馆、院系资料室被查阅,有权将学位论文的内容编入有关数据库进行检索,可以采用复印、缩印或其他方法保存学位论文。

\vspace*{40pt}
\begin{flushright}
\setlength{\@title@width}{5cm}
  {\sihao\song{
  \begin{tabular}{ll}
    学位论文作者签名: \qquad\qquad\qquad  &  导师签名: \qquad\qquad\qquad\\
    日期: \qquad 年\qquad 月\qquad 日     &  日期: \qquad 年\qquad 月\qquad 日 \\
  \end{tabular}}
 }
\end{flushright}
\thispagestyle{empty}   % 去掉页码

% 空白页
%\newpage
%\thispagestyle{empty}
%\mbox{}

%%%%%%%%%%%%%%%%%%% Abstract and Keywords 摘要和关键词 %%%%%%%%%%%%%%%%%%%%%%%

%中文摘要格式
\clearpage
\markboth{摘~要}{摘~要}
\pdfbookmark[0]{摘~~要}{cabstract}

% 摘要不加到目录中
%\addcontentsline{toc}{chapter}{摘要}

% 开始罗马数字编号
\setcounter{page}{1}
\pagenumbering{Roman}
\thispagestyle{plain}


\begin{flushleft}
\setlength{\@title@width}{5cm}
  {\xiaosi\song{
	\begin{tabular}{lcl}
	论文题目 & : & \@ctitle\\
	专业 & : & \@csubject \\
	博士生 & : & \@cauthor \\
	指导老师 & : & \@csupervisor \\
	\end{tabular}}
 }
\end{flushleft}



% 中文摘要:小二,黑体加粗,居中
\begin{center}
\xiaoer\hei\bfseries 摘\qquad 要
\end{center}

\vspace{\baselineskip} % 新增摘要后空行

% 插入中文摘要
\song\defaultfont
\@cabstract
\vspace{\baselineskip}

\hangafter=1\hangindent=52.3pt\noindent
{\hei\xiaosi 关键词:} \@ckeywords
%\thispagestyle{empty}

% 英文摘要格式
\clearpage
\markboth{ABSTRACT}{ABSTRACT}
\pdfbookmark[0]{ABSTRACT}{eabstract}

% 摘要不加到目录中
%\addcontentsline{toc}{chapter}{ABSTRACT}

\thispagestyle{plain}


% 如果英文title太长,手动分成两行
\begin{flushleft}
\setlength{\@title@width}{5cm}
  {\xiaosi{
  \begin{tabular}{ll}
	Title: & If the title is too long, break it into two lines manually \\
	& the longer part \\
    Major:      &  \@esubject \\
    Name:       &  \@eauthor \\
    Supervisor: &  \@esupervisor \\
  \end{tabular}}
 }
\end{flushleft}



% ABSTRACT三号居中
\begin{center}
\sanhao{\bf{ABSTRACT}}
\end{center}
\vspace{\baselineskip}

% 插入英文摘要
\@eabstract
\vspace{\baselineskip}

\hangafter=1\hangindent=60pt\noindent
{\textbf{Keywords:}} \@ekeywords
\thispagestyle{plain}

}
\makeatother
     

% ——————————————————————————————————————————————
% 以下是论文导言部分,包括论文的封面,中英文摘要和中文目录

\frontmatter
\fancypagestyle{plain}{
\fancyhf{}
\renewcommand{\headrulewidth}{0 pt}
\fancyfoot[C]{\song\xiaowu~\thepage~} %页脚
}

%%%%%%%%%%   封面   %%%%%%%%%%
% !Mode:: "TeX:UTF-8"

\cheading{中山大学博士学位论文开题报告}      % 设置正文的页眉,需要填上对应的毕业年份
\ctitle{xxxxxxxxxxxxxx}    % 封面用论文标题,自己可手动断行
\etitle{xxxxxxxxxxxxxx}    %论文英文标题
\csubject{xxxxxxxxx}   % 专业名称
\esubject{xxxxxxxxx}
\cauthor{x~~~~~~~~~~~~~~~x}            % 学生姓名
\eauthor{CAI Yun}
\csupervisor{xxx~~~~~~教授}        % 导师姓名,~用于间隔职称
\esupervisor{Prof. xxxxx}
\department{x~~~x~~~学~~~院}
\direction{xxxxxxxxx}

% 盲审时用
%\cauthor{XXX}            % 学生姓名
%\eauthor{XXX}
%\csupervisor{XXX}        % 导师姓名
%\esupervisor{Prof. XXX}

% 自动数字日期
\cdate{\the\year~年~\the\month~月~\the\day~日}
% 自动中文日期
%\cdate{\CJKnumber{\the\year}~年~\CJKnumber{\the\month}~月~\CJKnumber{\the\day}~日}
% 定制中文日期
%\cdate{二零一四~年~五~月~二十五~日}

\cabstract{
xxxxxxx
}

\ckeywords{xxx}

\eabstract{
xxxxxxx
}

\ekeywords{xxx}

\makecover

\clearpage


%%%%%%%%%%   目录   %%%%%%%%%%
\defaultfont
%\clearpage{\pagestyle{empty}\cleardoublepage}
\clearpage
%\pagestyle{empty}
%\setcounter{page}{1}                                 % 单独从 1 开始编页码
%\pagenumbering{arabic}
\titleformat{\chapter}{\centering\sanhao\hei}{\chaptername}{2em}{} % 设置目录两字的格式
\setcounter{tocdepth}{3}
\setcounter{secnumdepth}{4}
\pdfbookmark[0]{目~~~~录}{mulu}
\tableofcontents                                     % 中文目录
\fancypagestyle{plain}{
\fancyhf{}
\renewcommand{\headrulewidth}{0 pt}
\fancyfoot[C]{\song\xiaowu~\thepage~}
}
\renewcommand\listfigurename{插\ 图\ 目\ 录}
{%
\let\oldnumberline\numberline%
\renewcommand{\numberline}{\figurename~\oldnumberline}%
\listoffigures%
}
\renewcommand\listtablename{表\ 格\ 目\ 录}
{%
\let\oldnumberline\numberline%
\renewcommand{\numberline}{\tablename~\oldnumberline}%
\listoftables%
}
\thispagestyle{plain}

% ——————————————————————————————————————————————
% 以下是论文正文,内容在body子文件夹下

\mainmatter\defaultfont\sloppy\raggedbottom

\makeatletter
	\fancypagestyle{plain}{                              % 设置开章页眉页脚风格
		\fancyhf{}
		\fancyhead[C]{\song\wuhao \@cheading}            % 首页页眉格式
		\fancyfoot[C]{\song\xiaowu ~\thepage~}           % 首页页脚格式
		\renewcommand{\headrulewidth}{0.5pt}
		\renewcommand{\footrulewidth}{0pt}}
\makeatother

% 单独从 1 开始编页码
\setcounter{page}{1}
% chapter标题格式:小二,黑体,居中
\titleformat{\chapter}{\centering\sanhao\hei}{\chaptername}{2em}{}

%%%%%%%%%%   正文   %%%%%%%%%%

% !Mode:: "TeX:UTF-8"

\chapter{引言}
\label{ch:intro}

\section{研究背景}
\label{sec:background}
xxx
% !Mode:: "TeX:UTF-8"
\chapter{国内外研究现状}
\label{sec:review}
xxx
\section{相关问题综述}

xxx
% !Mode:: "TeX:UTF-8"

\chapter{第一章}
\section{问题描述}
% 术语(方法2)
\nomchn{CCDCCD}{charge-coupled device charge-coupled device charge-coupled device}{电荷耦合器件电荷耦合器件电荷耦合器件}
\nomchn{CCD}{charge-coupled device}{电荷耦合器件}

% 显示术语表
\printnomenclature[2cm]

% !Mode:: "TeX:UTF-8"

\chapter{第二章}
\section{问题描述}
xxxx
\include{body/chapter3}

%%%%%%%%%%  参考文献  %%%%%%%%%%
\lhead{} %页眉
\rhead{} %页脚
\chead{\song\wuhao 中山大学博士学位论文开题报告} % 覆盖设置页眉内容
\defaultfont
%\bibliographystyle{references/TJUThesis}
\phantomsection % 解决目录中超链接地址错误问题
\markboth{参考文献}{参考文献}
\addcontentsline{toc}{chapter}{参考文献}       % 参考文献加入到中文目录
\nocite{*}                                     % 若将此命令屏蔽掉,则未引用的文献不会出现在文后的参考文献中
\bibliography{references/thesis}
\bibliographystyle{apalike} 


% ——————————————————————————————————————————————
% 以下是论文附录,在appendix子文件下

% 重新设置想要的chapter格式
\titleformat{\chapter}{\centering\sihao\hei}{\chaptername}{2em}{}

% 攻读硕士学位期间发表学术论文情况
%% !Mode:: "TeX:UTF-8"

\markboth{攻读博士学位期间发表学术论文情况}{攻读博士学位期间发表学术论文情况}
%\addcontentsline{toc}{chapter}{攻读博士学位期间发表学术论文情况}
%\setcounter{page}{1}       % 如果需要从该页开始从 1 开始编页,则取消该注释
\chapter*{攻读博士学位期间发表学术论文情况}

\begin{enumerate}
	% 盲审时
	\item The paper title [J/C...]. Publish whereabout, 2014, 21(3): 288-291. (第一作者,导师为第一作者)(与学位论文第n章相关)(盲审时,不要出现名字)
	% 提交时
	\item Authors. The paper title [J/C...]. Publish whereabout, 2014, 21(3): 288-291. (提交时)
\end{enumerate}

% 仍然有页码
%\thispagestyle{empty}


% 致谢
%% !Mode:: "TeX:UTF-8"

\markboth{致\quad 谢}{致\quad 谢}
%\addcontentsline{toc}{chapter}{致\quad 谢} % 添加到目录中
\chapter*{致\quad 谢}



% 如果需要加名字和日期(日期根据生成文档日期变更)
\begin{flushright}
  \begin{tabular}{cl}
    李四 & \\
    二零一四年\CJKnumber{\the\month}月\CJKnumber{\the\day}日 & 
  \end{tabular}
\end{flushright}



\clearpage

% 结束中文字体使用
\end{CJK*}

% 结束全文
\end{document}